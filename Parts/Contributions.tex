\subsection{Choice of Topic}
We both really wanted to do something related to building, as we both are in robotics, and spend a lot of our free time making things. Alyssa really likes more hands on math, as it is easier to visualize what is going and understand the material. Not to mention, that we both think the images that are created are really spectacular.

When you look up harmonographs, often what comes up is Lissajous curves, which when dampening is taken into consideration, can describe the images made. Which is why we spent a lot of time researching Lissajous curves, and talk a lot about it. However, we also realized that due to the paramaterization of Lissajous curves, one can use line integrals to figure out the arc length of the image. Furthermore you can use a line integral to find the surface area of the image created.

\subsection{Paper}
\begin{itemize}
\item Initial Research -Alyssa
\item Rough Draft -Elias and Alyssa
\item History -Alyssa
\item Mathematical components -Elias
\item Arc Length and Surface Integral -Elias
\item Contributions -Alyssa
\item Conclusion -Alyssa
\end{itemize}
\subsection{Presentation}
\begin{itemize}
\item Rough Draft -Elias
\item History -Alyssa
\item Mathematical Components -Alyssa
\item Arc Length and Curvature -Elias
\item Our Harmonograph -Alyssa
\end{itemize}
\subsection{Harmonograph}
\begin{itemize}
\item Initial Research/Design -Alyssa
\item Materials -Alyssa
\item Built -Elias and Alyssa at Elias' shop
\item Transportation -Elias
\end{itemize}
